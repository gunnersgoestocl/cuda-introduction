\documentclass[dvipdfmx,11pt]{beamer}   % dvipdfmx で非 ASCII 画像も安全
\usetheme{metropolis}
\usecolortheme{default}
\title{CUDA: understand to be a genuine user}
\author{takigawa}
\date{\today}

%%%%%%%%%%%%%%%%%%%%%%%%%%%%%%%%%%%%%%%%
\begin{document}
%%%%%%%%%%%%%%%%%%%%%%%%%%%%%%
% タイトルスライド
\begin{frame}
  \titlepage       
\end{frame}
% 目次
\begin{frame}{Contents}
  \begin{enumerate}[<+->]   % <+-> で 1 行ずつ出現
    \item What is CUDA?: Introduction
    \item A CUDA program for beginners
    \item How CUDA works
    \item Optimize CUDA program
    \item Practice Problems
  \end{enumerate}
\end{frame}
%%%%%%%%%%%%%%%%%%%%%%%%%%%%%%
\section{What is CUDA?}
% 目次
\begin{frame}{Contents}
  \begin{enumerate}[<+->]   % <+-> で 1 行ずつ出現
    \item What is CUDA?: Introduction
    \item \textcolor{gray}{A CUDA program for beginners}
    \item \textcolor{gray}{How CUDA works}
    \item \textcolor{gray}{Optimize CUDA program}
    \item \textcolor{gray}{Practice Problems}
  \end{enumerate}
\end{frame}
%%%%%
\begin{frame}{CUDA is the abstraction of GPU(s) for programmers}
  
\end{frame}
%%%%%
\begin{frame}{To compile/run CUDA programs: NVCC}

\end{frame}
%%%%%%%%%%%%%%%%%%%%%%%%%%%%%%
\section{A CUDA program for beginners}
% 目次
\begin{frame}{Contents}
  \begin{enumerate}[<+->]   % <+-> で 1 行ずつ出現
    \item \textcolor{gray}{What is CUDA?: Introduction}
    \item A CUDA program for beginners
    \begin{enumerate}
      \item Kernels; writing and launching
      % __global__ kernelFunction, kernelFunction<<nb, bs>>(arg1, arg2, ...)
      \item Host-Device Data Communication (+ Synchronization)
      % cudaMalloc, cudaMemcpy, cudaMallocManaged, cudaDeviceSynchronize
    \end{enumerate}
    \item \textcolor{gray}{How CUDA works}
    \item \textcolor{gray}{Optimize CUDA program}
    \item \textcolor{gray}{Practice Problems}
  \end{enumerate}
\end{frame}
%%%%%%%%%
\begin{frame}{Setting environment: Using GH200(s) on miyabi}

\end{frame}
%%%%%%%%%
\begin{frame}{Sample program #1: \lstinline|hello_world.cu|}

\end{frame}
%%%%%
\begin{frame}{3 files are required for execution on miyabi}
% .cu, makefile, .sh

\end{frame}
%%%%%
\begin{frame}{Contents of .cu file: kernel function}
% keywords (__global__ etc.)
  
\end{frame}
%%%%%
\begin{frame}{Contents of .cu file: launching kernel by a host}
% specify the number of threads (and blocks), you can use dim3 struct

\end{frame}
%%%%%
\begin{frame}{Interlude: register values programs can explicitly use}
% gridDim, blockIdx, blockDim, threadIdx

\end{frame}
%%%%%%%%
\begin{frame}{Sample program #2: \lstinline|hello_gpu.cu|}

\end{frame}
%%%%%
\begin{frame}{Data communication between H & Ds}
% cudaMalloc, cudaMemcpy

\end{frame}
%%%%%
\begin{frame}{Host-Device synchronization}

\end{frame}
%%%%%
\begin{frame}{(+) "Programmer-free" data communication between H & Ds}
% cudaMallocManaged, Unified Memory

\end{frame}
%%%%%%%%%%%%%%%%%%%%%%%%%%%%%%
\section{HOW CUDA works}
% 目次
\begin{frame}{Contents}
  \begin{enumerate}[<+->]   % <+-> で 1 行ずつ出現
    \item \textcolor{gray}{What is CUDA?: Introduction}
    \item \textcolor{gray}{A CUDA program for beginners}
    \item How CUDA works
    \begin{enumerate}
      \item Architecture of NVIDIA GPU
      % GPU unit, global memory, stream processor
      \item Grid, block, thread; abstractions by CUDA
      % block-SM, grid-GPU unit
      \item Warp; Parallel Thread eXecution
      % warp consists of 32 threads, SIMD, why threadIdx/blockIdx
      \item Memory Hierarchy in CUDA
      % global memory, shared memory, cache
      \item Resolving race condition on CUDA
      % atomicAdd, barrier synchronization, reduction(, cooperative groups)
    \end{enumerate}
    \item \textcolor{gray}{Optimize CUDA program}
    \item \textcolor{gray}{Practice Problems}
  \end{enumerate}
\end{frame}
%%%%%%%%
\begin{frame}{Architecture of NVIDIA GPU: GPU unit}
% global memory, SM, link etc.

\end{frame}
%%%%%
\begin{frame}{Architecture of NVIDIA GPU: Stream Multiprocessors}
% CUDA core, regfile, Warp scheduler

\end{frame}
%%%%%
\begin{frame}{Architecture of NVIDIA GPU: Stream Multiprocessors}
% cache, shared memory

\end{frame}
%%%%%%%%
\begin{frame}{3 easy pieces about hardware - software}

\end{frame}
%%%%%
\begin{frame}{You may have questions like ...}
% 3 key questions to understand CUDA abstraction

\end{frame}
%%%%%%%%
\begin{frame}{Warp: The way to realize "Parallel" Thread eXecution}
  % Parallelism in your program is sometimes helped by Concurrency
\end{frame}
%%%%%
\begin{frame}{Hierarchy within an SM}
  % thread < warp < block < SM

\end{frame}
%%%%%
\begin{frame}{(+) Warp Scheduling}
  % GTO

\end{frame}
%%%%%%%%
\begin{frame}{Memory hierarchy of NVIDIA GPU}
  % global memory, shared memory, cache etc. Clarify which programmers can explicitly use

\end{frame}
%%%%%
\begin{frame}{Tiling: effective use of shared memory}

\end{frame}
%%%%%%%%
\begin{frame}{To avoid race conditions: effective use of parameters}

\end{frame}
%%%%%
\begin{frame}{To avoid race conditions: barrier synchronization}

\end{frame}
%%%%%
\begin{frame}{To avoid race conditions: Cooperative groups}

\end{frame}
%%%%%%%%
\begin{frame}{(Reluctantly) resolve race conditions: Atomic accumulations}

\end{frame}
%%%%%%%%%%%%%%%%%%%%%%%%%%%%%%
\section{Optimize CUDA program}
% 目次
\begin{frame}{Contents}
    \begin{enumerate}[<+->]   % <+-> で 1 行ずつ出現
    \item \textcolor{gray}{What is CUDA?: Introduction}
    \item \textcolor{gray}{A CUDA program for beginners}
    \item \textcolor{gray}{How CUDA works}
    \item Optimize CUDA program
    \begin{enumerate}
      \item How to measure performance?
      \item Choosing a good block size for performance
      \item Using shared memory effectively
      \item Example: matrix multiplication
    \end{enumerate}
    \item \textcolor{gray}{Practice Problems}
  \end{enumerate}
\end{frame}
%%%%%%%%
\begin{frame}{How to measure performance: }

\end{frame}
%%%%%
\begin{frame}{How to measure performance: }

\end{frame}
%%%%%%%%
\begin{frame}{Matmul program (naive implementation)}

\end{frame}
%%%%%%%%
\begin{frame}{Choosing a good block size, thread dim}

\end{frame}
%%%%%%%
\begin{frame}{Using shared memory, cache effectively}

\end{frame}
%%%%%%%
\begin{frame}{Optimized matmul program}

\end{frame}
%%%%%%%%%%%%%%%%%%%%%%%%%%%%%%
% 目次
\begin{frame}{Contents}
  \begin{enumerate}[<+->]   % <+-> で 1 行ずつ出現
    \item \textcolor{gray}{What is CUDA?: Introduction}
    \item \textcolor{gray}{A CUDA program for beginners}
    \item \textcolor{gray}{How CUDA works}
    \item \textcolor{gray}{Optimize CUDA program}
    \item Practice Problems
    \begin{enumerate}
      \item k-NN for vector Database
      \item SW's local alignment; DP acceleration with GPU
    \end{enumerate}
  \end{enumerate}
\end{frame}
%%%%%%%%
\begin{frame}{Problem #1: k-NN for vector Database}

\end{frame}
%%%%%
\begin{frame}{Hint: How to launch kernel across multiple GPUs}

\end{frame}
%%%%%%%%
\begin{frame}{Problem #2: local alignment of DNA arrays}

\end{frame}
%%%%%
\begin{frame}{Smith Water 's DP}

\end{frame}
%%%%%%%%
\begin{frame}{Further problems ...}
  % list of mundane problems

\end{frame}
%%%%%%%%%%%%%%%%%%%%%%%%%%%%%%
\begin{frame}{References}

\end{frame}
%%%%%%%%%%%%%%%%%%%%%%%%%%%%%%
\end{document}
